\documentclass[11pt]{article}
\usepackage[cache=false]{minted}
\usepackage{amsmath} %used for maths and "aligned" in equation
\usepackage{hyperref}


\usepackage{tcolorbox}
\usepackage{etoolbox}
\BeforeBeginEnvironment{minted}%
     {\begin{tcolorbox}}%
\AfterEndEnvironment{minted}
   {\end{tcolorbox}}%


\title{Algorithms}

\begin{document}
\maketitle

\tableofcontents

\section{Linked List}
We can use array to initialize the linked list as the following code.
While building up the linked list from the given array, we use the \textbf{two-pointer technique} to maintain the linking between two nodes.
\inputminted[breaklines=true,frame=leftline, linenos=true]{python}{src/linkedlist.py}

\section{Sliding Widow Technique}
\subsection{Count distinct elements in every window of size k}
Tag: Sliding Window Technique, Hashtable. 
See \footnote{https://www.geeksforgeeks.org/count-distinct-elements-in-every-window-of-size-k/}.

\begin{minted}[xleftmargin=20pt]{bash}
Input:  arr[] = {1, 2, 1, 3, 4, 2, 3}, k = 4
Output: [3, 4, 4, 3]
\end{minted}

We use the sliding window to update a hashtable, which maintains the distinct elements. And the time complexity is $O(n)$.

\inputminted{python}{src/distinct.py}

\subsection{Sliding Window Maximum (Maximum of all subarrays of size k)}

See \footnote{https://www.geeksforgeeks.org/sliding-window-maximum-maximum-of-all-subarrays-of-size-k/}.
\begin{minted}[xleftmargin=20pt]{bash}
Input :
arr[] = {1, 2, 3, 1, 4, 5, 2, 3, 6}
k = 3
Output :
3 3 4 5 5 5 6

Input :
arr[] = {8, 5, 10, 7, 9, 4, 15, 12, 90, 13}
k = 4
Output :
10 10 10 15 15 90 90
\end{minted}

We use the priority queue to .

\inputminted{python}{src/maxSlidingWindow.py}

\section{Heap}
Heap can be viewed as a complete tree, but stored as the array. 
Suppose the current node's index is $idx$, then the left child's index is $2*idx+1$, and the right child $2*idx+2$, while the parent $floor((idx-1)/2)$.

We take the binary max heap as an example. 
The basic external function is \textbf{insert} and \textbf{extractMax}, which is implemented by \textbf{siftup} and \textbf{siftdown}. 
The \textbf{siftup} function check the current node's value with its parent's value, then swap them if the current node's value is bigger than the parent's, and do the check-swap operation recursively to meet the guarantee 
	of the binary max heap.
	
The python source code is as following.
\inputminted[breaklines=true,frame=leftline, linenos=true]{python}{src/heap.py}

\subsection{Python's heapq}
We can use the library \textbf{heapq} in python. 
Since the default \textbf{heapq} is the min heap, so we need a trick to reimplement \textbf{MaxHeap} by overriding the comparison function. 
\inputminted[breaklines=true,frame=leftline, linenos=true]{python}{src/heap_heapq.py}

Or we can implement \textbf{MaxHeap} by multiplying -1 to each item in an array directly when using \textbf{heapq}.

\inputminted[breaklines=true,frame=leftline, linenos=true]{python}{src/heap_heapq2.py}

All the above three heap codes generate the following output. 
\begin{minted}[breaklines=true]{bash}
After inserting 1,2,3,4,5,6,7,8,9, the array of the heap is 9 8 6 7 3 2 5 1 4 .
Pop out from the heap, we'll get the maximum number 9, and the array of the heap becomes 8 7 6 4 3 2 5 1 .
\end{minted}

\subsection{The application of heap}
\subsubsection{Merge k Sorted Lists (LeetCode 23)}
Merge k sorted linked lists and return it as one sorted list. Analyze and describe its complexity\footnote{https://leetcode.com/problems/merge-k-sorted-lists/description/}.

\begin{minted}[xleftmargin=20pt]{bash}
Input:
[
  1->4->5,
  1->3->4,
  2->6
]
Output: 1->1->2->3->4->4->5->6
\end{minted}


\inputminted[breaklines=true,frame=leftline, linenos=true]{python}{src/mergeKSortedLists.py}

\subsubsection{Kth Largest Element in a Stream (LeetCode 703)}
Design a class to find the kth largest element in a stream. Note that it is the kth largest element in the sorted order, not the kth distinct element.

Your KthLargest class will have a constructor which accepts an integer k and an integer array nums, which contains initial elements from the stream. For each call to the method KthLargest.add, return the element representing the kth largest element in the stream.

Example:
\begin{minted}[xleftmargin=20pt]{bash}
int k = 3;
int[] arr = [4,5,8,2];
KthLargest kthLargest = new KthLargest(3, arr);
kthLargest.add(3);   // returns 4
kthLargest.add(5);   // returns 5
kthLargest.add(10);  // returns 5
kthLargest.add(9);   // returns 8
kthLargest.add(4);   // returns 8
\end{minted}

Note: 
\begin{enumerate}
	\item You may assume that nums' length $\geq k-1$  and $k \geq 1$.
\end{enumerate}

We use the min heap with the fixed size k to maintain the largest k elements in the stream. 
The minimum element in the min heap with size k will be the k-th largest element in a stream. 
The initialization of the heap is to heapify the given array.
Since the time complexity of the operation of "heapify" is $O(n)$\footnote{https://www.growingwiththeweb.com/data-structures/binary-heap/build-heap-proof/}, so the initialization is very efficient. 
And the add operation in the following code is to pop out the minimum element in the array which costs $O(\log k)$. 
In general, the time complexity is $O(n\log k)$, where $n$ is the size of the stream, and the space complexity is $O(\log k)$. 

\inputminted[breaklines=true,frame=leftline, linenos=true]{python}{src/kthLargestInStream.py}



\section{Dynamic Programming}

\section{Maths}
\subsection{Prime numbers}
\subsection{Count Primes (LeetCode 204)}
Count the number of prime numbers \textbf{less than} a non-negative number, n.
Example:
\begin{minted}[breaklines=true, xleftmargin=20pt]{bash}
Input: 10
Output: 4
Explanation: There are 4 prime numbers less than 10, they are 2, 3, 5, 7.
\end{minted}

Tag: Primes.

We use the \textbf{Sieve of Eratosthenes} to label each number within the array of $[1, \cdots, n]$ is a prime or not.
\inputminted[breaklines=true,frame=leftline, linenos=true]{python}{src/countPrimes.py}

The line 18 to line 19 in the code can be optimized as the following code without the time consumption on the loop. 
\begin{minted}[breaklines=true, xleftmargin=20pt]{python}
primeFlags[p*p:n:p] = [False]*len(primeFlags[p*p:n:p])	
\end{minted}

By eliminating the inner loop, the time consumption is reduced from 860ms to 304ms. 

\subsubsection{Ugly Number (LeetCode 263)}
Write a program to check whether a given number is an ugly number.

Ugly numbers are positive numbers whose prime factors only include 2, 3, 5.

Example 1:
\begin{minted}[breaklines=true, xleftmargin=20pt]{bash}
Input: 6
Output: true
Explanation: 6 = 2 * 3
\end{minted}

Example 2:
\begin{minted}[breaklines=true, xleftmargin=20pt]{bash}
Input: 8
Output: true
Explanation: 8 = 2 * 2 * 2
\end{minted}

Example 3:
\begin{minted}[breaklines=true, xleftmargin=20pt]{bash}
Input: 14
Output: false 
Explanation: 14 is not ugly since it includes another prime factor 7.
\end{minted}

Note:
\begin{enumerate}
	\item 1 is typically treated as an ugly number.
	\item Input is within the 32-bit signed integer range: [$-2^{31}$, $2^{31}-1$].
\end{enumerate}

We use the while loop to do the check and the decomposition for a given number. Since the given number is within the range [$-2^{31}$, $2^{31}-1$], so we can do the check and the decomposition by recursion without worrying about the stack overflow (exceeding the maximum recursion depth). 
\inputminted[breaklines=true,frame=leftline, linenos=true]{python}{src/isUgly.py}

The recursion version is as following. 
\inputminted[breaklines=true,frame=leftline, linenos=true]{python}{src/isUgly_recursion.py}

\subsection{Ugly Number II (LeetCode 264)}
Write a program to find the n-th ugly number. 
Ugly numbers are positive numbers whose prime factors only include 2, 3, 5. 
Example:
\begin{minted}[breaklines=true, xleftmargin=20pt]{bash}
Input: n = 10
Output: 12
Explanation: 1, 2, 3, 4, 5, 6, 8, 9, 10, 12 is the sequence of the first 10 ugly numbers.
\end{minted}
Note:  
\begin{enumerate}
\item 1 is typically treated as an ugly number.
\item n does not exceed 1690.
\end{enumerate}

Tag: Maths, Primes, Tricky.

This problem's solution is very very tricky.
Although the tag on this problem includes dynamic programming. 
But I don't think it's a good example of the dynamic programming technique, because it can not get a clear \textbf{recursive formula}.
Instead, I would rather call it a tricky solution only using a tabulation. 

Suppose the resulted ugly number list is $F$. 
Since F includes the numbers whose factors only include 2, 3, and 5. 
So we build up three lists $l_2=2*F$, $l_3=3*F$, and $l_5=5*F$.
Therefore the ugly list $F$ has the property that $F=[1]+ merge(l_2,l_3,l_5)$.
Based on this property, what we should do is merging $l_2$, $l_3$, and $l_5$. 
The \textbf{tricky part} is that $F$ should be merged from $l_2$, $l_3$, and $l_5$, while these three lists are also need to be built from $F$.
We handle it by updating them simultaneously as the following code. 

\inputminted[breaklines=true,frame=leftline, linenos=true]{python}{src/uglyNumberII.py}

If we want to save the space of $l_2$, $l_3$, and $l_5$, then we'll do not claim space for them but use the space of $F$ only by maintaining the pointers in these three lists. The code is as following.
\inputminted[breaklines=true,frame=leftline, linenos=true]{python}{src/uglyNumberII_singleList.py}

%\begin{equation}
%	\begin{aligned}
%		F(n)=
%		\left\{ \begin{aligned}
%		1 &, n=1 & (a)\\
%		2 &, n=2 & (b)\\
%		3 &, n=3 & (c)\\
%		4 &, n=4 & (d)\\
%		5 &, n=5 & (e)\\
%		min\{2*F(n-3), 3*F(n-3), 5*F(n-3)\} &, n>5 & (f)
%		\end{aligned}\right.
%	\end{aligned}
%\end{equation}

\subsection{Super Ugly Number (LeetCode 313)}
Write a program to find the nth super ugly number.

Super ugly numbers are positive numbers whose all prime factors are in the given prime list primes of size k.

Example:
\begin{minted}[breaklines=true, xleftmargin=20pt]{bash}
Input: n = 12, primes = [2,7,13,19]
Output: 32 
Explanation: [1,2,4,7,8,13,14,16,19,26,28,32] is the sequence of the first 12 
             super ugly numbers given primes = [2,7,13,19] of size 4.
\end{minted}
Note:
\begin{enumerate}
	\item 1 is a super ugly number for any given primes.
	\item The given numbers in primes are in ascending order.
	\item $0 < k \leq 100$, $0 < n \leq 10^6$, $0 < \text{primes[i]} < 1000$.
	\item The nth super ugly number is guaranteed to fit in a 32-bit signed integer.
\end{enumerate}

Our solution is treat this problem as the extension of the problem \textbf{Ugly Number II}. 
This method's time complexity is $O(nk)$, where $k$ is the length of the array primes. 
But this method is not optimized, which should be speeded up to $O(n log k)$. 
Think about the problem \textbf{Merging k Sorted Lists}, which is optimized by using the data structure heap. 

\inputminted[breaklines=true,frame=leftline, linenos=true]{python}{src/nthSuperUglyNumber.py}


\end{document}
