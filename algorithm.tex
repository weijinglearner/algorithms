\documentclass[11pt]{article}
\usepackage[cache=false]{minted}
\usepackage{hyperref}


\usepackage{tcolorbox}
\usepackage{etoolbox}
\BeforeBeginEnvironment{minted}%
     {\begin{tcolorbox}}%
\AfterEndEnvironment{minted}
   {\end{tcolorbox}}%


\title{Algorithms}

\begin{document}
\maketitle

\section{Sliding Widow Technique}
\subsection{Count distinct elements in every window of size k}
Tag: Sliding Window Technique, Hashtable. 
See \footnote{https://www.geeksforgeeks.org/count-distinct-elements-in-every-window-of-size-k/}.

\begin{minted}[xleftmargin=20pt]{bash}
Input:  arr[] = {1, 2, 1, 3, 4, 2, 3}, k = 4
Output: [3, 4, 4, 3]
\end{minted}

We use the sliding window to update a hashtable, which maintains the distinct elements. And the time complexity is $O(n)$.

\inputminted{python}{src/distinct.py}

\subsection{Sliding Window Maximum (Maximum of all subarrays of size k)}

\inputminted{python}{src/maxSlidingWindow.py}
	
\end{document}
